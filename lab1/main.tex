\documentclass[a4paper,oneside,14pt]{extarticle}

\include{preamble}

\begin{document}

\include{title}
\setcounter{page}{2}
\renewcommand{\contentsname}{СОДЕРЖАНИЕ}
\tableofcontents

\newpage

\section{Цели}

Основной целью работы является ознакомление с принципами функционирования, построения и особенностями архитектуры суперскалярных конвейерных микропроцессоров.

Дополнительной целью работы является знакомство с принципами проектирования и верификации сложных цифровых устройств с использованием языка описания аппаратуры SystemVerilog и ПЛИС.

\section{Задания}

Далее представлен ход выполнения заданий лабораторной работы.

\subsection{Задание 1}

Далее представлены листинги, полученные в процессе выполнения первого задания лабораторной работы.

\begin{code}
\begin{lstinputlisting}[
        label={lst:1},
        caption={Код программы по варианту на языке ассемблера}
    ]{lst/1.s}
\end{lstinputlisting}
\end{code}

\begin{code}
\begin{lstinputlisting}[
        label={lst:1},
        caption={Дизассемблированный листинг кода программы по варианту}
    ]{lst/text.s}
\end{lstinputlisting}
\end{code}

\begin{code}
\begin{lstinputlisting}[
        label={lst:1},
        caption={Код программы по варианту в шестнадцатеричном представлении}
    ]{lst/1.hex}
\end{lstinputlisting}
\end{code}

\begin{code}
\begin{lstinputlisting}[
        label={lst:1},
        language={C++},
        caption={Псевдокод программы по варианту}
    ]{lst/1.c}
\end{lstinputlisting}
\end{code}

В результате выполнения программы, в регистре \texttt{x31} будет содержаться число, соответствующее максимальному элементу массива \texttt{x\_}, то есть \texttt{8}.

\subsection{Задание 2}

\begin{code}
\begin{lstinputlisting}[
        label={lst:1},
        caption={Дизассемблированный листинг кода тестовой программы test.s}
    ]{lst/test.s}
\end{lstinputlisting}
\end{code}

\begin{figure}[H]
	\centering
	\includegraphics[width=1\textwidth]{img/2.png}
	\caption{Временная диаграмма выполнения стадий выборки и диспечеризации команды по адресу 80000024, 2-я итерация}
	\label{fig:2}
\end{figure}

\subsection{Задание 3}

\begin{figure}[H]
	\centering
	\includegraphics[width=1\textwidth]{img/3.png}
    \caption{Временная диаграмма выполнения стадии декодирования и планирования на выполнение команды по адресу 80000030, 2-я итерация}
	\label{fig:3}
\end{figure}

\subsection{Задание 4}

\begin{figure}[H]
	\centering
	\includegraphics[width=1\textwidth]{img/4.png}
    \caption{Временная диаграмма выполнения стадии выполнения команды по адресу 8000001c, 2-я итерация}
	\label{fig:4}
\end{figure}

\subsection{Задание 5}

\begin{figure}[H]
	\centering
	\includegraphics[width=1\textwidth]{img/5.png}
    \caption{Временная диаграмма всех стадий выполнения команды, обозначенной в тексте программы \#!}
	\label{fig:5}
\end{figure}

\begin{figure}[H]
	\centering
	\includegraphics[width=1\textwidth]{img/x31.png}
    \caption{В результате выполнения программы, регистр x31 принимает значение 8, как и ожидалось}
	\label{fig:x31}
\end{figure}

\begin{figure}[H]
	\centering
	\includegraphics[width=1\textwidth]{img/pipeline-unoptimized.pdf}
    \caption{Трасса выполнения неоптимизированной программы}
	\label{fig:pu}
\end{figure}

Конфликты возникают из-за того, что происходит обращение к регистру, запись в который ещё не была завершена.
Как можно заметить по псевдокоду, строка x1 += enroll (add x1, x1, elem\_sz*enroll) будет выполнена вне зависимости от предшествующих условных переходов, а также значение регистра x1 не используется после строки lw x3, 4(x1) в пределах выполнения тела цикла.
Следовательно, если поместить строку add x1, x1, slem\_sz*enroll сразу после lw x3, 4(x1), это не повлияет на результат выполнения программы, зато даст дополнительную задержку в один такт между загрузкой значения в x2 и чтением из него, что позволит устранить конфликты.

\begin{code}
\begin{lstinputlisting}[
        label={lst:1},
        caption={Оптимизированный код программы по варианту на языке ассемблера}
    ]{lst/opt.s}
\end{lstinputlisting}
\end{code}

\begin{code}
\begin{lstinputlisting}[
        label={lst:1},
        caption={Дизассемблированный листинг оптимизированного кода программы по варианту}
    ]{lst/opt-text.s}
\end{lstinputlisting}
\end{code}

\begin{code}
\begin{lstinputlisting}[
        label={lst:1},
        caption={Оптимизированный код программы по варианту в шестнадцатеричном представлении}
    ]{lst/opt.hex}
\end{lstinputlisting}
\end{code}

\begin{code}
\begin{lstinputlisting}[
        label={lst:1},
        language={C++},
        caption={Псевдокод оптимизированной программы по варианту}
    ]{lst/opt.c}
\end{lstinputlisting}
\end{code}

\begin{figure}[H]
	\centering
	\includegraphics[width=1\textwidth]{img/pipeline-optimized.pdf}
    \caption{Трасса выполнения оптимизированной программы}
	\label{fig:po}
\end{figure}

\newpage

\section{Заключение}

В результате данной работы были изучены принципы функционирования, построения и особенности архитектуры суперскалярных конвейерных микропроцессоров.

На основе изученных материалов был найден способ оптимизировать программу.

Цель работы была достигнута.

\end{document}
