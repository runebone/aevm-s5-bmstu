\documentclass[a4paper,oneside,14pt]{extarticle}

\include{preamble}

\begin{document}

\include{title}
\setcounter{page}{2}
\renewcommand{\contentsname}{СОДЕРЖАНИЕ}
\tableofcontents

\newpage

\section{Задание 1}

% \begin{code}
\begin{lstinputlisting}[
        label={lst:1},
        caption={}
    ]{lst/lab1.txt}
\end{lstinputlisting}
% \end{code}

\section{Задание 2}

% \begin{code}
\begin{lstinputlisting}[
        label={lst:1},
        caption={}
    ]{lst/lab2.txt}
\end{lstinputlisting}
% \end{code}

Запись в логе:

% \begin{code}
\begin{lstinputlisting}[
        label={lst:1},
        caption={}
    ]{lst/lab2log.txt}
\end{lstinputlisting}
% \end{code}

\newpage

\section{Индивидуальное задание}

\subsection{Условие}
\textbf{Устройство формирования индексов SQL UNION.}
Сформировать в хост-подсистеме и передать в SPE 256 записей множества A (случайные числа в диапазоне 0..1024) и 256 записей множества B (случайные числа в диапазоне 0..1024).
Сформировать в SPE множество C = A or B.
Выполнить тестирование работы SPE, сравнив набор ключей в множестве C с ожидаемым.

\subsection{Изменённый файл host\_main.cpp}

% \begin{code}
\begin{lstinputlisting}[
        label={lst:1},
        caption={}
    ]{lst/host_main.cpp}
\end{lstinputlisting}
% \end{code}

\subsection{Изменённый файл common\_struct.h}

% \begin{code}
\begin{lstinputlisting}[
        label={lst:1},
        caption={}
    ]{lst/common_struct.h}
\end{lstinputlisting}
% \end{code}

\subsection{Изменённый файл sw\_kernel\_main.cpp}

% \begin{code}
\begin{lstinputlisting}[
        label={lst:1},
        caption={}
    ]{lst/sw_kernel_main.cpp}
\end{lstinputlisting}
% \end{code}

\subsection{Результат работы программы}

% \begin{code}
\begin{lstinputlisting}[
        label={lst:1},
        caption={}
    ]{lst/mainlog.txt}
\end{lstinputlisting}
% \end{code}

% \newpage

\phantomsection\section*{ЗАКЛЮЧЕНИЕ}\addcontentsline{toc}{section}{ЗАКЛЮЧЕНИЕ}

В результате работы изучены принципы работы вычислительного комплекса Тераграф и получены практические навыки решения задач обработки множеств на основе гетерогенной вычислительной структуры. 

\end{document}
