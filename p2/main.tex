\documentclass[a4paper,oneside,14pt]{extarticle}

\include{preamble}

\begin{document}

\include{title}
\setcounter{page}{2}
\renewcommand{\contentsname}{СОДЕРЖАНИЕ}
\tableofcontents

\newpage

\section{Задание}

\begin{figure}[H]
	\centering
	\includegraphics[width=0.9\textwidth]{img/boogiewoogie-newmann.pdf}
	\caption{shuffle\_boogie\_woogie\_with\_trad\_changes\_C\_major.pcl3 --- Визуализация на основе модулярности Ньюмана}
	\label{fig:}
\end{figure}

\begin{figure}[H]
	\centering
	\includegraphics[width=0.9\textwidth]{img/boogiewoogie-lattice.pdf}
    \caption{shuffle\_boogie\_woogie\_with\_trad\_changes\_C\_major.pcl3 --- Визуализация графа-решётки на основе центральности}
	\label{fig:}
\end{figure}

\begin{figure}[H]
	\centering
	\includegraphics[width=0.9\textwidth]{img/harrypotter-newmann.pdf}
	\caption{HarryPotterPrologue\_A\_minor.pcl3 --- Визуализация на основе модулярности Ньюмана}
	\label{fig:}
\end{figure}

\begin{figure}[H]
	\centering
	\includegraphics[width=0.9\textwidth]{img/harrypotter-lattice.pdf}
	\caption{HarryPotterPrologue\_A\_minor.pcl3 --- Визуализация графа-решётки на основе центральности}
	\label{fig:}
\end{figure}

\begin{figure}[H]
	\centering
	\includegraphics[width=0.9\textwidth]{img/metallica-newmann.pdf}
	\caption{Metallica\_-\_The\_Unforgiven\_A\_minor.pcl3 --- Визуализация на основе модулярности Ньюмана}
	\label{fig:}
\end{figure}

\begin{figure}[H]
	\centering
	\includegraphics[width=0.9\textwidth]{img/up-lattice.pdf}
	\caption{Upmaintheme\_C\_major.pcl3 --- Визуализация графа-решётки на основе центральности}
	\label{fig:}
\end{figure}

\section{Дополнительное задание}

\begin{figure}[H]
	\centering
	\includegraphics[width=0.9\textwidth]{img/source-pic.png}
	\caption{Исходное изображение}
	\label{fig:}
\end{figure}

\begin{figure}[H]
	\centering
	\includegraphics[width=0.9\textwidth]{img/boogiewoogie-newmann.pdf}
	\caption{Граф, который будет совмещён с исходным изображением}
	\label{fig:}
\end{figure}

\begin{figure}[H]
	\centering
    \includegraphics[width=0.9\textwidth]{img/combine.png}
	\caption{Результат совмещения графа с исходным изображением}
	\label{fig:}
\end{figure}

\section{Времена года}

\begin{figure}[H]
	\centering
    \includegraphics[width=0.9\textwidth]{img/dalle1.png}
	\caption{Ноябрь, версия 1}
	\label{fig:}
\end{figure}

% \begin{figure}[H]
% 	\centering
%     \includegraphics[width=0.9\textwidth]{img/dalle2.png}
% 	\caption{Ноябрь, версия 2}
% 	\label{fig:}
% \end{figure}

\begin{figure}[H]
	\centering
    \includegraphics[width=0.9\textwidth]{img/graphs-fly-to-the-south-1.png}
	\caption{Ноябрь, версия 2}
	\label{fig:}
\end{figure}

\begin{figure}[H]
	\centering
    \includegraphics[width=0.9\textwidth]{img/graphs-fly-to-the-south-2.png}
	\caption{Ноябрь, версия 3}
	\label{fig:}
\end{figure}

% \phantomsection\section*{ЗАКЛЮЧЕНИЕ}\addcontentsline{toc}{section}{ЗАКЛЮЧЕНИЕ}

\end{document}
