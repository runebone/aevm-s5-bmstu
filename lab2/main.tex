\documentclass[a4paper,oneside,14pt]{extarticle}

\include{preamble}

\begin{document}

\include{title}
\setcounter{page}{2}
\renewcommand{\contentsname}{СОДЕРЖАНИЕ}
\tableofcontents

\newpage

\phantomsection\section*{ВВЕДЕНИЕ}\addcontentsline{toc}{section}{ВВЕДЕНИЕ}

Цель работы --- освоение принципов эффективного использования подсистемы памяти современных универсальных ЭВМ, обеспечивающей хранение и своевременную выдачу команд и данных в центральное процессорное устройство.
Работа проводится с использованием программы для сбора и анализа производительности PCLAB.

\newpage

\section{Эксперимент 1: Исследование расслоения динамической памяти}

\subsection{Параметры эксперимента}

\begin{enumerate}
    \item Параметр 1 (максимальное расстояние между читаемыми данными): 64
    \item Параметр 2 (шаг увеличения расстояния между читаемыми 4х байтовыми ячейками): 64
    \item Параметр 3 (размер массива): 8
\end{enumerate}

\subsection{Результат эксперимента}

\begin{figure}[H]
	\centering
	\includegraphics[width=1\textwidth]{img/1.png}
    \caption{Исследование расслоения динамической памяти}
	\label{fig:1}
\end{figure}

% \subsection{Цель эксперимента}

% \subsection{Описание проблемы}

% \subsection{Суть эксперимента}

% \subsection{Условия эксперимента}

% \subsection{Результаты эксперимента}

% \subsection{Вывод}

\newpage

\section{Эксперимент 2: Сравнение эффективности ссылочных и векторных структур}

\subsection{Параметры эксперимента}

\begin{enumerate}
    \item Параметр 1 (количество элементов в списке): 1
    \item Параметр 2 (максимальная фрагментация списка): 256
    \item Параметр 3 (шаг изменения фрагментации): 4
\end{enumerate}

\subsection{Результат эксперимента}

\begin{figure}[H]
	\centering
	\includegraphics[width=1\textwidth]{img/2.png}
    \caption{Сравнение эффективности ссылочных и векторных структур}
	\label{fig:2}
\end{figure}

Список обрабатывался в 40.391009 раз дольше.

% \subsection{Цель эксперимента}

% \subsection{Описание проблемы}

% \subsection{Суть эксперимента}

% \subsection{Условия эксперимента}

% \subsection{Результаты эксперимента}

% \subsection{Вывод}

\newpage

\section{Эксперимент 3: Исследование эффективности программной предвыборки}

\subsection{Параметры эксперимента}

\begin{enumerate}
    \item Параметр 1 (шаг увеличения расстояния между читаемыми данными): 256
    \item Параметр 2 (размер массива): 512
\end{enumerate}

\subsection{Результат эксперимента}

\begin{figure}[H]
	\centering
	\includegraphics[width=1\textwidth]{img/3.png}
    \caption{Исследование эффективности программной предвыборки}
	\label{fig:3}
\end{figure}

Обработка без загрузки таблицы страниц в TLB производилась в 1.347938 раз дольше.

% \subsection{Цель эксперимента}

% \subsection{Описание проблемы}

% \subsection{Суть эксперимента}

% \subsection{Условия эксперимента}

% \subsection{Результаты эксперимента}

% \subsection{Вывод}

\newpage

\section{Эксперимент 4: Исследование способов эффективного чтения оперативной памяти}

\subsection{Параметры эксперимента}

\begin{enumerate}
    \item Параметр 1 (размер массива): 4
    \item Параметр 2 (количество потоков данных): 64
\end{enumerate}

\subsection{Результат эксперимента}

\begin{figure}[H]
	\centering
	\includegraphics[width=1\textwidth]{img/4.png}
    \caption{Исследование способов эффективного чтения оперативной памяти}
	\label{fig:4}
\end{figure}

Неоптимизированная структура обрабатывалась в 1.1646418 раз дольше.

% \subsection{Цель эксперимента}

% \subsection{Описание проблемы}

% \subsection{Суть эксперимента}

% \subsection{Условия эксперимента}

% \subsection{Результаты эксперимента}

% \subsection{Вывод}

\newpage

\section{Эксперимент 5: Исследование конфликтов в кеш-памяти}

\subsection{Параметры эксперимента}

\begin{enumerate}
    \item Параметр 1 (размер блока кеш-памяти): 256
    \item Параметр 2 (размер линейки кеш-памяти): 128
    \item Параметр 3 (количество читаемых линеек): 512
\end{enumerate}

\subsection{Результат эксперимента}

\begin{figure}[H]
	\centering
	\includegraphics[width=1\textwidth]{img/5.png}
    \caption{Исследование конфликтов в кеш-памяти}
	\label{fig:5}
\end{figure}

Чтение с конфликтами банков производилось в 1.0824 раз дольше.

% \subsection{Цель эксперимента}

% \subsection{Описание проблемы}

% \subsection{Суть эксперимента}

% \subsection{Условия эксперимента}

% \subsection{Результаты эксперимента}

% \subsection{Вывод}

\newpage

\section{Эксперимент 6: Сравнение алгоритмов сортировки}

\subsection{Параметры эксперимента}

\begin{enumerate}
    \item Параметр 1 (количество 64-х разрядных элементов массивов): 1
    \item Параметр 2 (шаг увеличения размера массива): 4
\end{enumerate}

\subsection{Результат эксперимента}

\begin{figure}[H]
	\centering
	\includegraphics[width=1\textwidth]{img/6.png}
    \caption{Сравнение алгоритмов сортировки}
	\label{fig:6}
\end{figure}

QuickSort работал в 1.6806711 раз дольше Radix-Counting Sort.
QuickSort работал в 1.7781202 раз дольше Radix-Counting Sort, оптимизированного под 8-процессорную ЭВМ.

% \subsection{Цель эксперимента}

% \subsection{Описание проблемы}

% \subsection{Суть эксперимента}

% \subsection{Условия эксперимента}

% \subsection{Результаты эксперимента}

% \subsection{Вывод}

% \begin{figure}[H]
% 	\centering
% 	\includegraphics[width=1\textwidth]{img/pipeline-optimized.pdf}
%     \caption{Трасса выполнения оптимизированной программы}
% 	\label{fig:po}
% \end{figure}

\newpage

\phantomsection\section*{ЗАКЛЮЧЕНИЕ}\addcontentsline{toc}{section}{ЗАКЛЮЧЕНИЕ}

Освоены принципы эффективного использования подсистемы памяти современных универсальных ЭВМ, обеспечивающей хранение и своевременную выдачу команд и данных в центральное процессорное устройство.
Работа была проведена с использованием программы для сбора и анализа производительности PCLAB.
Поставленная цель достигнута.

\end{document}
